% Options for packages loaded elsewhere
\PassOptionsToPackage{unicode}{hyperref}
\PassOptionsToPackage{hyphens}{url}
\PassOptionsToPackage{dvipsnames,svgnames,x11names}{xcolor}
%
\documentclass[
  letterpaper,
  DIV=11,
  numbers=noendperiod]{scrreprt}

\usepackage{amsmath,amssymb}
\usepackage{iftex}
\ifPDFTeX
  \usepackage[T1]{fontenc}
  \usepackage[utf8]{inputenc}
  \usepackage{textcomp} % provide euro and other symbols
\else % if luatex or xetex
  \usepackage{unicode-math}
  \defaultfontfeatures{Scale=MatchLowercase}
  \defaultfontfeatures[\rmfamily]{Ligatures=TeX,Scale=1}
\fi
\usepackage{lmodern}
\ifPDFTeX\else  
    % xetex/luatex font selection
\fi
% Use upquote if available, for straight quotes in verbatim environments
\IfFileExists{upquote.sty}{\usepackage{upquote}}{}
\IfFileExists{microtype.sty}{% use microtype if available
  \usepackage[]{microtype}
  \UseMicrotypeSet[protrusion]{basicmath} % disable protrusion for tt fonts
}{}
\makeatletter
\@ifundefined{KOMAClassName}{% if non-KOMA class
  \IfFileExists{parskip.sty}{%
    \usepackage{parskip}
  }{% else
    \setlength{\parindent}{0pt}
    \setlength{\parskip}{6pt plus 2pt minus 1pt}}
}{% if KOMA class
  \KOMAoptions{parskip=half}}
\makeatother
\usepackage{xcolor}
\setlength{\emergencystretch}{3em} % prevent overfull lines
\setcounter{secnumdepth}{5}
% Make \paragraph and \subparagraph free-standing
\ifx\paragraph\undefined\else
  \let\oldparagraph\paragraph
  \renewcommand{\paragraph}[1]{\oldparagraph{#1}\mbox{}}
\fi
\ifx\subparagraph\undefined\else
  \let\oldsubparagraph\subparagraph
  \renewcommand{\subparagraph}[1]{\oldsubparagraph{#1}\mbox{}}
\fi

\usepackage{color}
\usepackage{fancyvrb}
\newcommand{\VerbBar}{|}
\newcommand{\VERB}{\Verb[commandchars=\\\{\}]}
\DefineVerbatimEnvironment{Highlighting}{Verbatim}{commandchars=\\\{\}}
% Add ',fontsize=\small' for more characters per line
\usepackage{framed}
\definecolor{shadecolor}{RGB}{241,243,245}
\newenvironment{Shaded}{\begin{snugshade}}{\end{snugshade}}
\newcommand{\AlertTok}[1]{\textcolor[rgb]{0.68,0.00,0.00}{#1}}
\newcommand{\AnnotationTok}[1]{\textcolor[rgb]{0.37,0.37,0.37}{#1}}
\newcommand{\AttributeTok}[1]{\textcolor[rgb]{0.40,0.45,0.13}{#1}}
\newcommand{\BaseNTok}[1]{\textcolor[rgb]{0.68,0.00,0.00}{#1}}
\newcommand{\BuiltInTok}[1]{\textcolor[rgb]{0.00,0.23,0.31}{#1}}
\newcommand{\CharTok}[1]{\textcolor[rgb]{0.13,0.47,0.30}{#1}}
\newcommand{\CommentTok}[1]{\textcolor[rgb]{0.37,0.37,0.37}{#1}}
\newcommand{\CommentVarTok}[1]{\textcolor[rgb]{0.37,0.37,0.37}{\textit{#1}}}
\newcommand{\ConstantTok}[1]{\textcolor[rgb]{0.56,0.35,0.01}{#1}}
\newcommand{\ControlFlowTok}[1]{\textcolor[rgb]{0.00,0.23,0.31}{#1}}
\newcommand{\DataTypeTok}[1]{\textcolor[rgb]{0.68,0.00,0.00}{#1}}
\newcommand{\DecValTok}[1]{\textcolor[rgb]{0.68,0.00,0.00}{#1}}
\newcommand{\DocumentationTok}[1]{\textcolor[rgb]{0.37,0.37,0.37}{\textit{#1}}}
\newcommand{\ErrorTok}[1]{\textcolor[rgb]{0.68,0.00,0.00}{#1}}
\newcommand{\ExtensionTok}[1]{\textcolor[rgb]{0.00,0.23,0.31}{#1}}
\newcommand{\FloatTok}[1]{\textcolor[rgb]{0.68,0.00,0.00}{#1}}
\newcommand{\FunctionTok}[1]{\textcolor[rgb]{0.28,0.35,0.67}{#1}}
\newcommand{\ImportTok}[1]{\textcolor[rgb]{0.00,0.46,0.62}{#1}}
\newcommand{\InformationTok}[1]{\textcolor[rgb]{0.37,0.37,0.37}{#1}}
\newcommand{\KeywordTok}[1]{\textcolor[rgb]{0.00,0.23,0.31}{#1}}
\newcommand{\NormalTok}[1]{\textcolor[rgb]{0.00,0.23,0.31}{#1}}
\newcommand{\OperatorTok}[1]{\textcolor[rgb]{0.37,0.37,0.37}{#1}}
\newcommand{\OtherTok}[1]{\textcolor[rgb]{0.00,0.23,0.31}{#1}}
\newcommand{\PreprocessorTok}[1]{\textcolor[rgb]{0.68,0.00,0.00}{#1}}
\newcommand{\RegionMarkerTok}[1]{\textcolor[rgb]{0.00,0.23,0.31}{#1}}
\newcommand{\SpecialCharTok}[1]{\textcolor[rgb]{0.37,0.37,0.37}{#1}}
\newcommand{\SpecialStringTok}[1]{\textcolor[rgb]{0.13,0.47,0.30}{#1}}
\newcommand{\StringTok}[1]{\textcolor[rgb]{0.13,0.47,0.30}{#1}}
\newcommand{\VariableTok}[1]{\textcolor[rgb]{0.07,0.07,0.07}{#1}}
\newcommand{\VerbatimStringTok}[1]{\textcolor[rgb]{0.13,0.47,0.30}{#1}}
\newcommand{\WarningTok}[1]{\textcolor[rgb]{0.37,0.37,0.37}{\textit{#1}}}

\providecommand{\tightlist}{%
  \setlength{\itemsep}{0pt}\setlength{\parskip}{0pt}}\usepackage{longtable,booktabs,array}
\usepackage{calc} % for calculating minipage widths
% Correct order of tables after \paragraph or \subparagraph
\usepackage{etoolbox}
\makeatletter
\patchcmd\longtable{\par}{\if@noskipsec\mbox{}\fi\par}{}{}
\makeatother
% Allow footnotes in longtable head/foot
\IfFileExists{footnotehyper.sty}{\usepackage{footnotehyper}}{\usepackage{footnote}}
\makesavenoteenv{longtable}
\usepackage{graphicx}
\makeatletter
\def\maxwidth{\ifdim\Gin@nat@width>\linewidth\linewidth\else\Gin@nat@width\fi}
\def\maxheight{\ifdim\Gin@nat@height>\textheight\textheight\else\Gin@nat@height\fi}
\makeatother
% Scale images if necessary, so that they will not overflow the page
% margins by default, and it is still possible to overwrite the defaults
% using explicit options in \includegraphics[width, height, ...]{}
\setkeys{Gin}{width=\maxwidth,height=\maxheight,keepaspectratio}
% Set default figure placement to htbp
\makeatletter
\def\fps@figure{htbp}
\makeatother

\KOMAoption{captions}{tableheading}
\makeatletter
\@ifpackageloaded{bookmark}{}{\usepackage{bookmark}}
\makeatother
\makeatletter
\@ifpackageloaded{caption}{}{\usepackage{caption}}
\AtBeginDocument{%
\ifdefined\contentsname
  \renewcommand*\contentsname{Table of contents}
\else
  \newcommand\contentsname{Table of contents}
\fi
\ifdefined\listfigurename
  \renewcommand*\listfigurename{List of Figures}
\else
  \newcommand\listfigurename{List of Figures}
\fi
\ifdefined\listtablename
  \renewcommand*\listtablename{List of Tables}
\else
  \newcommand\listtablename{List of Tables}
\fi
\ifdefined\figurename
  \renewcommand*\figurename{Figure}
\else
  \newcommand\figurename{Figure}
\fi
\ifdefined\tablename
  \renewcommand*\tablename{Table}
\else
  \newcommand\tablename{Table}
\fi
}
\@ifpackageloaded{float}{}{\usepackage{float}}
\floatstyle{ruled}
\@ifundefined{c@chapter}{\newfloat{codelisting}{h}{lop}}{\newfloat{codelisting}{h}{lop}[chapter]}
\floatname{codelisting}{Listing}
\newcommand*\listoflistings{\listof{codelisting}{List of Listings}}
\makeatother
\makeatletter
\makeatother
\makeatletter
\@ifpackageloaded{caption}{}{\usepackage{caption}}
\@ifpackageloaded{subcaption}{}{\usepackage{subcaption}}
\makeatother
\ifLuaTeX
  \usepackage{selnolig}  % disable illegal ligatures
\fi
\usepackage{bookmark}

\IfFileExists{xurl.sty}{\usepackage{xurl}}{} % add URL line breaks if available
\urlstyle{same} % disable monospaced font for URLs
\hypersetup{
  pdftitle={Introdução à análise espacial em saúde no R},
  pdfauthor={Mário Círio Nogueira},
  colorlinks=true,
  linkcolor={blue},
  filecolor={Maroon},
  citecolor={Blue},
  urlcolor={Blue},
  pdfcreator={LaTeX via pandoc}}

\title{Introdução à análise espacial em saúde no R}
\author{Mário Círio Nogueira}
\date{Invalid Date}

\begin{document}
\maketitle

\renewcommand*\contentsname{Table of contents}
{
\hypersetup{linkcolor=}
\setcounter{tocdepth}{2}
\tableofcontents
}
\bookmarksetup{startatroot}

\chapter{Introdução à análise espacial em saúde no
R}\label{introduuxe7uxe3o-uxe0-anuxe1lise-espacial-em-sauxfade-no-r}

\bookmarksetup{startatroot}

\chapter{Introdução}\label{introduuxe7uxe3o}

Bem-vindo ao meu ebook. Este é um livro criado com o \texttt{R} e
\textbf{Quarto}.

\bookmarksetup{startatroot}

\chapter{Epidemiologia}\label{epidemiologia}

\begin{center}\rule{0.5\linewidth}{0.5pt}\end{center}

\begin{center}\rule{0.5\linewidth}{0.5pt}\end{center}

A \textbf{Epidemiologia} é o estudo da distribuição das doenças e
problemas de saúde em populações humanas (epidemiologia descritiva) e
das causas desta distribuição (epidemiologia analítica). A distribuição
das doenças pode ser analisada nas dimensões temporal, espacial e
pessoal. Em relação à dimensão temporal, queremos saber se a frequência
de determinada doença está diminuindo, estável ou aumentando com o
passar do tempo. Os estudos utilizam dados agregados e são chamados
estudos de tendência/série temporal. Podemos também estar interessados
em variações sazonais (estações do ano ou dias da semana) ou em
variações diárias, que acompanham variações em outros indicadores, por
exemplo climáticos. Quanto à dimensão pessoal, procuramos identificar
características dos indivíduos associadas à maior frequência da doença.
Neste caso estão os estudos com dados individualizados: inquéritos,
coortes, caso-controle e ensaios clínicos. Por fim, de maior interesse
para este livro, está a investigação da distribuição espacial das
doenças. Os estudos utilizam dados agregados por recortes geográficos
das populações (setores censitários, bairros, municípios, regiões,
estados, países). Estes estudos são chamados de ecológicos, agregados ou
de análise espacial em saúde.

\bookmarksetup{startatroot}

\chapter{R: linguagem e programa}\label{r-linguagem-e-programa}

\begin{center}\rule{0.5\linewidth}{0.5pt}\end{center}

\begin{center}\rule{0.5\linewidth}{0.5pt}\end{center}

O R é um programa estatístico de domínio público criado em 1995 a partir
da linguagem S. Ele roda em Windows, Mac ou Linux e sua versão mais
recente pode ser baixada gratuitamente do
\href{https://cran.r-project.org/}{CRAN - Comprehensive R Archive
Network}. As suas funções estatísticas são ativadas pela digitação de
comandos; estes podem ser arquivados em arquivos textos chamados
scripts, facilitando muito seu uso.\\
O RStudio é uma plataforma integrada de trabalho que torna mais fácil a
manipulação do R e também pode ser baixada gratuitamente da internet
(\href{https://rstudio.com/}{R Studio}). Tem que instalar primeiro o R e
depois o RStudio.\\
O \href{http://www.r-project.org/}{Projeto R} é uma colaboração
internacional de pesquisadores de diversas áreas, como estatística,
ciências da computação, epidemiologia, geografia, etc, que desenvolvem
novos conjuntos de funções no R, chamados pacotes ou bibliotecas, e que
também podem ser baixados gratuitamente da internet. Diversos manuais e
tutoriais estão disponíveis na internet, alguns acessados a partir do
\href{https://cran.r-project.org/}{CRAN}.

\bookmarksetup{startatroot}

\chapter{Objetos no R}\label{objetos-no-r}

\begin{center}\rule{0.5\linewidth}{0.5pt}\end{center}

\begin{center}\rule{0.5\linewidth}{0.5pt}\end{center}

O R tem 5 classes básicas (atômicas) de objetos:\\
- character (exemplo: ``a'')\\
- numeric: (exemplo: 3.5)\\
- integer: (exemplo: 3)\\
- logical: (exemplo: TRUE)\\
- complex: (exemplo: (1 + 3i))\\
A função class() informa qual a classe do objeto.

\begin{Shaded}
\begin{Highlighting}[]
\NormalTok{x }\OtherTok{\textless{}{-}} \StringTok{"a"}  
\FunctionTok{class}\NormalTok{(x)}
\end{Highlighting}
\end{Shaded}

\begin{verbatim}
[1] "character"
\end{verbatim}

\begin{Shaded}
\begin{Highlighting}[]
\NormalTok{x }\OtherTok{\textless{}{-}} \FloatTok{3.5}  
\FunctionTok{class}\NormalTok{(x)}
\end{Highlighting}
\end{Shaded}

\begin{verbatim}
[1] "numeric"
\end{verbatim}

\begin{Shaded}
\begin{Highlighting}[]
\NormalTok{x }\OtherTok{\textless{}{-}} \DecValTok{3}  
\FunctionTok{class}\NormalTok{(x)}
\end{Highlighting}
\end{Shaded}

\begin{verbatim}
[1] "numeric"
\end{verbatim}

\begin{Shaded}
\begin{Highlighting}[]
\NormalTok{x }\OtherTok{\textless{}{-}} \FunctionTok{as.integer}\NormalTok{(x)}
\FunctionTok{class}\NormalTok{(x)}
\end{Highlighting}
\end{Shaded}

\begin{verbatim}
[1] "integer"
\end{verbatim}

\begin{Shaded}
\begin{Highlighting}[]
\NormalTok{x }\OtherTok{\textless{}{-}} \ConstantTok{TRUE}  
\FunctionTok{class}\NormalTok{(x)}
\end{Highlighting}
\end{Shaded}

\begin{verbatim}
[1] "logical"
\end{verbatim}

\begin{Shaded}
\begin{Highlighting}[]
\NormalTok{x }\OtherTok{\textless{}{-}}\NormalTok{ (}\DecValTok{1} \SpecialCharTok{+} \DecValTok{3}\NormalTok{i)  }
\FunctionTok{class}\NormalTok{(x)}
\end{Highlighting}
\end{Shaded}

\begin{verbatim}
[1] "complex"
\end{verbatim}

Outros objetos armazenam os objetos atômicos:\\
* Vetor: sequência de objetos atômicos de mesma classe (ex.:
c(1,2,3)).\\
* Fator: vetor de inteiros com rótulos/levels.\\
* Matriz: vetor com duas dimensões (linhas e colunas); os objetos
atômicos também são todos da mesma classe.\\
* Lista: conjunto de vetores, matrizes e/ou dataframes; seus elementos
não precisam ter o mesmo comprimento; geralmente os resultados das
funções são armazenados em listas.\\
* Dataframe: lista especial na qual todos os elementos têm o mesmo
comprimento; os elementos podem ser de tipos básicos diferentes;
geralmente os bancos de dados são armazenados como dataframes, nos quais
os registros são as linhas e as variáveis são as colunas.\\
* Função: conjunto de comandos que vão executar ordens específicas.
Muitas funções já vêm nos pacotes básicos do R (instalados juntamente
com o programa), várias outras são ativadas quando habilitamos
pacotes/bibliotecas do R específicas, e também podemos criar as funções
que precisarmos.

Quando temos dúvidas sobre alguma função do R, temos várias opções de
ajuda:\\
* Documentação do R: help(``nome\_da\_função'') ou ?nome\_da\_função.\\
* Digitar no Google: ``r project função''.\\
* Digitar no \href{https://stackoverflow.com/}{stackoverflow}: ``r
project função''.

Podemos também pesquisar pelos pacotes do R no
\href{https://cran.r-project.org/}{CRAN}, no Google ou no
\href{https://stackoverflow.com/}{stackoverflow}.

Para instalar um pacote no R (apenas na primeira vez que for usar)
usa-se a seguinte função: install.packages(``nome do pacote''). Para
ativar o pacote a cada sessão do R: library(``nome do pacote''). Para
saber quais pacotes estão ativados no momento: (.packages()).

\bookmarksetup{startatroot}

\chapter{Dataframe}\label{dataframe}

É um objeto muito importante do R porque geralmente nossos bancos de
dados são armazenados como dataframes. São semelhantes a planilhas, em
que as linhas correspondem a registros/observações e as colunas a
variáveis.\\
Algumas funções úteis para lidar com os dataframes:\\
* dim() - Número de linhas (registros) e de colunas (variáveis)\\
* head() - Mostra as primeiras 6 linhas\\
* tail() - Mostra as últimas 6 linhas\\
* names() - Os nomes das colunas\\
* str() - Estrutura do data.frame. Mostra, entre outras coisas, as
classes de cada coluna\\
* cbind() - Acopla duas tabelas lado a lado\\
* rbind() - Empilha duas tabelas

\bookmarksetup{startatroot}

\chapter{Objetos no R}\label{objetos-no-r-1}

O R tem 5 classes básicas (atômicas) de objetos:\\
- character (exemplo: ``a'')\\
- numeric: (exemplo: 3.5)\\
- integer: (exemplo: 3)\\
- logical: (exemplo: TRUE)\\
- complex: (exemplo: (1 + 3i))\\
A função class() informa qual a classe do objeto.

\begin{Shaded}
\begin{Highlighting}[]
\NormalTok{x }\OtherTok{\textless{}{-}} \StringTok{"a"}  
\FunctionTok{class}\NormalTok{(x)}
\end{Highlighting}
\end{Shaded}

\begin{verbatim}
[1] "character"
\end{verbatim}

\begin{Shaded}
\begin{Highlighting}[]
\NormalTok{x }\OtherTok{\textless{}{-}} \FloatTok{3.5}  
\FunctionTok{class}\NormalTok{(x)}
\end{Highlighting}
\end{Shaded}

\begin{verbatim}
[1] "numeric"
\end{verbatim}

\begin{Shaded}
\begin{Highlighting}[]
\NormalTok{x }\OtherTok{\textless{}{-}} \DecValTok{3}  
\FunctionTok{class}\NormalTok{(x)}
\end{Highlighting}
\end{Shaded}

\begin{verbatim}
[1] "numeric"
\end{verbatim}

\begin{Shaded}
\begin{Highlighting}[]
\NormalTok{x }\OtherTok{\textless{}{-}} \FunctionTok{as.integer}\NormalTok{(x)}
\FunctionTok{class}\NormalTok{(x)}
\end{Highlighting}
\end{Shaded}

\begin{verbatim}
[1] "integer"
\end{verbatim}

\begin{Shaded}
\begin{Highlighting}[]
\NormalTok{x }\OtherTok{\textless{}{-}} \ConstantTok{TRUE}  
\FunctionTok{class}\NormalTok{(x)}
\end{Highlighting}
\end{Shaded}

\begin{verbatim}
[1] "logical"
\end{verbatim}

\begin{Shaded}
\begin{Highlighting}[]
\NormalTok{x }\OtherTok{\textless{}{-}}\NormalTok{ (}\DecValTok{1} \SpecialCharTok{+} \DecValTok{3}\NormalTok{i)  }
\FunctionTok{class}\NormalTok{(x)}
\end{Highlighting}
\end{Shaded}

\begin{verbatim}
[1] "complex"
\end{verbatim}

Outros objetos armazenam os objetos atômicos:\\
* Vetor: sequência de objetos atômicos de mesma classe (ex.:
c(1,2,3)).\\
* Fator: vetor de inteiros com rótulos/levels.\\
* Matriz: vetor com duas dimensões (linhas e colunas); os objetos
atômicos também são todos da mesma classe.\\
* Lista: conjunto de vetores, matrizes e/ou dataframes; seus elementos
não precisam ter o mesmo comprimento; geralmente os resultados das
funções são armazenados em listas.\\
* Dataframe: lista especial na qual todos os elementos têm o mesmo
comprimento; os elementos podem ser de tipos básicos diferentes;
geralmente os bancos de dados são armazenados como dataframes, nos quais
os registros são as linhas e as variáveis são as colunas.\\
* Função: conjunto de comandos que vão executar ordens específicas.
Muitas funções já vêm nos pacotes básicos do R (instalados juntamente
com o programa), várias outras são ativadas quando habilitamos
pacotes/bibliotecas do R específicas, e também podemos criar as funções
que precisarmos.

Quando temos dúvidas sobre alguma função do R, temos várias opções de
ajuda:\\
* Documentação do R: help(``nome\_da\_função'') ou ?nome\_da\_função.\\
* Digitar no Google: ``r project função''.\\
* Digitar no \href{https://stackoverflow.com/}{stackoverflow}: ``r
project função''.

Podemos também pesquisar pelos pacotes do R no
\href{https://cran.r-project.org/}{CRAN}, no Google ou no
\href{https://stackoverflow.com/}{stackoverflow}.

Para instalar um pacote no R (apenas na primeira vez que for usar)
usa-se a seguinte função: install.packages(``nome do pacote''). Para
ativar o pacote a cada sessão do R: library(``nome do pacote''). Para
saber quais pacotes estão ativados no momento: (.packages()).

\bookmarksetup{startatroot}

\chapter{Dataframe}\label{dataframe-1}

É um objeto muito importante do R porque geralmente nossos bancos de
dados são armazenados como dataframes. São semelhantes a planilhas, em
que as linhas correspondem a registros/observações e as colunas a
variáveis.\\
Algumas funções úteis para lidar com os dataframes:\\
* dim() - Número de linhas (registros) e de colunas (variáveis)\\
* head() - Mostra as primeiras 6 linhas\\
* tail() - Mostra as últimas 6 linhas\\
* names() - Os nomes das colunas\\
* str() - Estrutura do data.frame. Mostra, entre outras coisas, as
classes de cada coluna\\
* cbind() - Acopla duas tabelas lado a lado\\
* rbind() - Empilha duas tabelas

\bookmarksetup{startatroot}

\chapter{Operações simples}\label{operauxe7uxf5es-simples}

\begin{center}\rule{0.5\linewidth}{0.5pt}\end{center}

\begin{center}\rule{0.5\linewidth}{0.5pt}\end{center}

Podemos executar no R desde as operações matemáticas mais simples até os
modelos estatísticos mais complexos. Divisão e multiplicação são
calculadas antes da adição e subtração, e podemos usar parênteses para
especificar a ordem em que as operações devem ser feitas. Os códigos
para as operações matemáticas simples são:\\
* Adição: +\\
* Subtração: -\\
* Multiplicação: *\\
* Divisão: /\\
* Potência: \^{}\\
* Raiz quadrada: sqrt()

\begin{Shaded}
\begin{Highlighting}[]
\DecValTok{2}\SpecialCharTok{+}\DecValTok{2}
\end{Highlighting}
\end{Shaded}

\begin{verbatim}
[1] 4
\end{verbatim}

\begin{Shaded}
\begin{Highlighting}[]
\DecValTok{3{-}1}
\end{Highlighting}
\end{Shaded}

\begin{verbatim}
[1] 2
\end{verbatim}

\begin{Shaded}
\begin{Highlighting}[]
\DecValTok{3}\SpecialCharTok{*}\DecValTok{3}
\end{Highlighting}
\end{Shaded}

\begin{verbatim}
[1] 9
\end{verbatim}

\begin{Shaded}
\begin{Highlighting}[]
\DecValTok{4}\SpecialCharTok{/}\DecValTok{4}
\end{Highlighting}
\end{Shaded}

\begin{verbatim}
[1] 1
\end{verbatim}

\begin{Shaded}
\begin{Highlighting}[]
\DecValTok{3}\SpecialCharTok{\^{}}\DecValTok{2}
\end{Highlighting}
\end{Shaded}

\begin{verbatim}
[1] 9
\end{verbatim}

\begin{Shaded}
\begin{Highlighting}[]
\FunctionTok{sqrt}\NormalTok{(}\DecValTok{9}\NormalTok{)}
\end{Highlighting}
\end{Shaded}

\begin{verbatim}
[1] 3
\end{verbatim}

\bookmarksetup{startatroot}

\chapter{Entrada de dados no R}\label{entrada-de-dados-no-r}

\begin{center}\rule{0.5\linewidth}{0.5pt}\end{center}

\begin{center}\rule{0.5\linewidth}{0.5pt}\end{center}

A forma mais prática de leitura de bancos de dados no R é no formato de
arquivos texto (csv: comma separeted values). É importante saber o que
está sendo usado para separar as colunas (``,'' ou ``;'' ou ``/'') e
qual o separador de decimais (``.'' ou ``,''). Se temos um banco de
dados no formato de planilha, podemos exportar o arquivo como csv para a
importação pelo R.

\section{Usando o pacote base}\label{usando-o-pacote-base}

A função mais flexível é read.table, em que podemos especificar as
características acima. Exemplo (lendo o arquivo e salvando como um
objeto dataframe no R):\\
* df \textless- read.table(``nomedoarquivo.csv'', header=T, sep=``;'',
dec=``.'')

\section{Usando o tidyverse}\label{usando-o-tidyverse}

O tidyverse tem um pacote para leitura e importação de bancos de dados.



\end{document}
